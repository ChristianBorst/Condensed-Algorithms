\documentclass{article}

\usepackage{amsmath}
\usepackage{amssymb}
\usepackage{mathtools}
\usepackage{fullpage}
\usepackage{enumerate}
\usepackage{graphicx}

\title{Computer Science 577 Notes \\ Introduction to Algorithms}
\author{Mendel C. Mayr}
\date{\today}

\begin{document}
	\maketitle
	\vspace{10pt}
	\begin{center}
		\includegraphics[width = 2.7in]{mergesort.png}
		\end{center}
	\vspace{12pt}
	\tableofcontents
	\clearpage

	\section{Recurrence Relations and Analysis of Algorithms}
		\subsection{Recurrence Relations}
			\subsubsection{Recursive Analysis of Insertion and Merge Sort}
				Insertion sort: let $M(n)$ be the comparisons required to sort a list of size $n$ \\
				\\
				Analysis: note that $M(1) = 0$ and $M(n) = M(n - 1) + n$ for $n > 1$
				\begin{enumerate}[(i)]
					\item $M(n) = M(n - 1) + n$
					\item $M(n) = M(n - 2) + n + (n - 1)\:...$
					\item $M(n) = M(n - k) + n + (n - 1) +\:...\:+ (n - k + 1)$
					\item Let $k = n - 1$, $M(n) = M(1) + n(n - n + 1) + \sum_{i = 1}^{n - 1}i$
					\item $M(n) = 0 + n + (n - 1)(n - 2)/2 \approx n^2/2$
					\end{enumerate}
				Merge sort: let $M(n)$ be the comparison required to sort a list of size $n$ \\
				\\
				Analysis: for simplcity, consider only $n$ such that $n = 2^a$ for some integer $a$ \\
				Note that $M(1) = 0$ and $M(n) = 2M(n/2) + n$ for $n > 1$
				\begin{enumerate}[(i)]
				 	\item $M(n) = 2M(n/2) + n$
				 	\item $M(n) = 2M(n/4) + n + (n/2)$
				 	\item $M(n) = 2M(n/2^k) + n + (n/2) +\:....\:+ (n/2^{k - 1})$
				 	\item Let $k = a$, $2M(1) + \sum_{i = 1}^{k - 1} n/2^i$ (unclarified point)
				 	\item Mergsort is $O(n \log n)$
				 	\end{enumerate}
			\subsubsection{Recursive Linear Selection}
				Recursive linear selection algorithm: given $x_1, x_2, ..., x_n$ distinct keys, find $x_k$ (i.e. the $k$th smallest element) without using sorting \\
				Note: the rank of an element (i.e. the number of keys greater than it) can be found in linear time \\
				Linear selection algorithm is as folows:
				\begin{enumerate}[(i)]
					\item Remove keys of known rank, to make $n = 5 (mod 10)$
					\item Divide elements into groups of 5, denoted $S[i]$ for $i$ from $1$ to $n/5$
					\item Recursively find the median of each group, denoted $x[i]$
					\item Let $M^*$ be the median of the set $x[i]$ for $i$ from $1$ to $n/5$
					\item Divide keys into groups of keys less than (call this $L$), equal to, or greater than (call this $R$) $M^*$
					\item Recursiveley process one of $L$ or $R$
					\end{enumerate}
				Analysis: Note that steps 1, 2, and 5 are $O(n)$, so the number of computations for this algorithm, $T(n) = T(n/5) + T(7n/10) + O(n)$ \\
				Guessing and proving: supposed that $T(n) = O(n)$, which can be proven via strong induction, i.e. $T(n) < An$ for some constant $A$ for all $n$ \\
				Recall that strong induction relies on proving two claims:
				\begin{enumerate}[(i)]
					\item The statement holds for all $n \geq 1$, $\forall n \leq n_0$ (base case)
					\item If the statmenet holds for all $i < n$, it holds for $n$
					\end{enumerate}
				Proof of second part of strong inductive proof:
				\begin{enumerate}[(i)]
					\item Suppose that $T(n) = O(n)$ for $i < n$
					\item We seek and $A$ such that $A(n/5) + A(7n/10) + cn \leq An$
					\item Thus, $A \geq 10c$ is sufficient for this part
					\end{enumerate}
				Proof of first part of strong inductive proof: need $A$ such that $n(n - 1)/2 \leq An$ for $1 \leq n \leq 10$ \\
				So $A \geq 9/10$ is sufficient for this part \\
				\\
				Conclusion: $A = max\{9/2, 10c\}$ will suffice to show that $T(n) = O(n)$ 
			\subsubsection{Recursive Quadratic Closest-Pair}
				Recursive quadratic algorithm: find closest pair of points
				\begin{enumerate}[i]
					\item (Supposing that $n = 2^k$) into 2 equal groups, denoted $L$ and $R$
					\item Recurisvely find the closest pair in $L$ and $R$
					\item Report closest pair form testing elements of $L$ against elements of $R$
					\item Report best pair out of those from steps (ii) and (iii)
					\end{enumerate}
				Analysis: $T(n) = 2(T/n) + O(n^2)$ for $n = 2^k \geq 4$, $T(2) = 1$ \\
				The $O(n^2)$ in the recursive case comes from step (iii) \\
				\\
				Consider the recursion tree, which is full binary tree: at the first level, the problem size is $n, n/2, n/4, ...$ at the first, second, third, etc. levels. Thus, the number of computations required is $n^2$ at the first level, $2(n/2)^2 = n^2/2$ at the second level, $2(n/4)^2 = n^2/4$ at the third, etc. \\
				\\
				Thus, the maximum number of computations is $\sum_{k = 1}^\infty n^2/2^k = 2n^2$, thus $T(n) = O(n^2)$
			\subsubsection{Divide and Conquer Recurrences}
				Master theorem: if $T(n) = aT(n/b) + O(n^d)$ for some constants $a > 0, b > 1$, and $d \geq 0$, then:
				\begin{enumerate}[(i)]
					\item $T(n) = O(n^d)$ if $d > \log_b a$
					\item $T(n) = O(n^d \log n)$ if $d = \log_b a$
					\item $T(n) = O(n^{\log_b a})$ if $d < \log_b a$
					\end{enumerate}
				Proof: consider the recursion tree for such a problem \\
				Notice that $a$ is the branching factor of the problem. At the $i$th level (starting at index 0), there are $a^i$ subproblems of size $n/b^i$. which means the computation that must be done at that level is $a^iO((n/b^i)^d)$ \\
				The number of levels in the recusion tree is $k = \log_b n$ \\
				As such: $T(n) = \sum_{i = 0}^k a_iO((n/b^i)^d) = O(n^d)\sum_{i = 0}^k(a/b^d)^i$. Now consider the cases
				\begin{enumerate}[(i)]
					\item If $d > \log_b a$, then $a/b^d < 1$ \\
					$\sum_{i = 0}^\infty (a/b^d)^i < \infty$ (i.e. series converges) \\
					$\sum_{i = 0}^\infty (a/b^d)^i = O(1)$, so $T(n) = O(n^d)$
					\item If $d = \log_b a$, then $a/b^d = 1$ \\
					$\sum_{i = 0}^k (a/b^d)^i = \sum_{i = 0}^k 1 = k + 1$, since $k = \log_b n = \log n/\log b = O(\log n)$ \\
					Therefore, $T(n) = O(n^d\log n)$
					\item If $d < \log_b a$, then $a/b^d > 1$ \\
					$\sum_{i = 0}^k (a/b^d)^i = O((a/b^d)^k)$, so $T(n) = O(n^d)O(a^k)/b^{dk}$ \\
					Since $k = \log_b n, n = b^k$, $T(n) = O(n^d)O(a^k)/n^d = O(a^k)$ \\
					$a^k = a^{\log_b n} = n^{\log_b a}$, so $T(n) = O(n^{\log_b a})$
					\end{enumerate}

	\appendix

	\section{Review of Basic Mathematical Concepts}
		\subsection{Review of Properties of Logarithms}
			Change of base: $\log_a x = \log_b x/\log_b a$ \\
			Basic properties:
			\begin{enumerate}[(i)]
				\item $\log_a(uv) = \log_a u + \log_a v$
				\item $\log_a(u / v) = \log_a u - \log_a v$
				\item $\log_a u^n = n \log_a u$
				\end{enumerate}



	\end{document}